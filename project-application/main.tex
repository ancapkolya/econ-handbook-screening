\documentclass{report}
\usepackage{graphicx}
\usepackage[dvipsnames]{xcolor}
\usepackage[most]{tcolorbox}
\usepackage{amsmath}
\usepackage{pgfplots}
\usepgfplotslibrary{fillbetween}
\pgfplotsset{width=10cm,compat=1.9}

\usepackage[letterpaper,top=2cm,bottom=2cm,left=2cm,right=2cm,marginparwidth=1.75cm]{geometry}
\setlength\parindent{15pt}

\usepackage[unicode]{hyperref}
\hypersetup{colorlinks,
  allcolors=[RGB]{010 090 200}}

\usepackage[T2A]{fontenc}
\usepackage[russian]{babel}
\usepackage{indentfirst}
\renewcommand{\rmdefault}{cmss}
\renewcommand{\ttdefault}{cmss}

\newcommand{\mycolor}[1]{RoyalBlue!#1!white}
\newcommand{\mycolorframe}{100}
\newcommand{\mycolorlower}{100}
\newcommand{\mycolorupper}{100}
\newcommand{\mycolorback}{10}
\newcommand{\mycolortext}{black}

\newcommand{\invitelink}{https://t.me/+t6oKRvp2l6oxMWQ6}

\newtcolorbox{mybox}[1]{breakable,colupper=\mycolortext,colframe=\mycolor{\mycolorframe},colback=\mycolor{\mycolorback},collower=\mycolortext,fonttitle=\bfseries,title={#1}}
\newtcolorbox{mybox_2}{breakable,colupper=\mycolortext,colframe=\mycolor{\mycolorframe},colback=\mycolor{\mycolorback},collower=\mycolortext,fonttitle=\bfseries}

%\let\oldtableofcontents\tableofcontents
%\renewcommand{\tableofcontents}{
%\begin{tcolorbox}[breakable,colback=\mycolor{\mycolorback},colframe=\mycolor{\mycolorframe},collower=\mycolor{\mycolorlower},colupper=\mycolor{\mycolorupper}]
%    \oldtableofcontents
%\end{tcolorbox}}


\title{Проектная заявка}
\author{Тумаков Николай}
\date{\today}


\begin{document}

\maketitle

\section*{\centering{\textit{\textbf{\href{https://github.com/ancapkolya/econ-handbook-screening}{github страница
проекта}}}}}

\section{Название}
Пособие по олимпиадной экономике по темам: "Модель Хотеллинга. Сигналинг на рынке труда. Вертикальная дифференциация товара"

\section{Целевая аудитория}
Олимпиадники по экономике, изучившие основную теорию, которые хотят изучить более сложные темы. Возраст: 14-17 лет.

\section{Проблемное поле}
В своем пособии я хочу разобрать несколько продвинутых тем, которые часто вызывают затруднения у олимпиадников и которые сложно с нуля самостоятельно изучить. Обычно этих тем нет в открытом доступе. Поэтому я считаю и подкрепляю свое заявление результатами опроса на выборке 25 из мотивированных олимпиадников (\href{https://docs.google.com/document/d/1WK_wnkck95BfUkFfhx7ZT7HdcT96Bs2Y3aL5qas-0lE/edit?usp=sharing}{ссылка}), что многим ученикам пособие по этим темам может быть полезно. В своем пособии я хочу объединить несколько моделей, которые могут учитывать какую-то степень качества объекта, ниже поясню.

\begin{enumerate}
    \item Модель Хотеллинга Одна из самых сложных тем в олимпиадной экономике в том числе среди других
    олигополистических взаимодействий. Так как фирме кроме выбранного количества/цены конкурента надо еще учитывать
    ее расположение, а также  транзакционные издержки покупателей. Данная модель часто встречается на олимпиадах в
    различных вариациях. В том числе несколько раз задачи на эту тему попадались на заключительном этапе ВСОШ и на
    втором этапе Высшей Пробы
    \item Сигналинг на рынке труда Это тема еще не попадалась на олимпиадах,
    тем не менее она встречается в различных выездных школах, например
    на интенсиве ВЭШ к ВП и на выездной школе олмата к региональному
    этапу. Модель демонстрирует взаимодействие работодателя и работника, в
    зависимости от уровня образования работника.
    \item Вертикальная дифференциация товара Эта тема тоже нова для школьной олимпиадной экономики. Она появилась в
    первый раз на заключительном этапе всош 2022 года. Поэтому в будущем она может попастся и на других олимпиадах.
    Эта модель подразумевает выбор качества товара на рынке олигополии.
\end{enumerate}

\section{Почему я выбрал именно эти темы?}
Многие темы уже подробно раскрыты и описаны в них. Поэтому я решил выбрать темы, которые точно не заняты, интересны
мне и которые сложно изучить в открытых источниках. Такое пособие может быть полезно ученикам, которые по тем или
иным причинам не попали на курсы выездных школ, чтобы понизить им порог входа для изучения этих тем. А также потому
что многие онлайн-школы заявляют, что в олимпиадах появляются более "ВУЗовские" модели, поэтому важно дать
представление об этих не очень известных темах.

\section{Что объединяет эти модели?}
В этих моделях мы учитываем некий дополнительный, неценовой фактор, как бы еще одну переменную, которую агенты могут
учитывать при выборе своего оптимального поведения. В модели Хотеллинга - это расположение потребителей. В модели
Вертикальной дифференциации товара - это качество товара, которое наблюдают потребители и готовы заплатить разную
цену в зависимости от этого. В отличие от Хотеллинга потребители разделяются не по местоположению, а по уровню
достатка. В модели Спенса про сигналинг на рынке труда, кажется, что происходит то же самое. Только товар (свой труд
) предоставляют соискатели работы, они как-будто выбирают свое качество (в данном случае - уровень образования), а
работодатель (который решил все с конца) предлагает им соответствующий контракт и в будущем получает разную отдачу от
"умных" и обычных работников. Во всех этих моделях важна концепция безразличного потребителя, которому оба продукта
приносят одинаковую полезность. Во всех этих моделях одна группа агентов при взаимодействии с другой разделяются в
зависимости от некоторой категориальной переменной.

\section{Образ продукта}
\subsection{Оформление}
Мое пособие будет выполнено в системе компьютерной верстки LaTeX с использованием пакета tcolorbox.
\subsection{Формат}
\begin{enumerate}
    \item Несколько вспомогательных тем
    \item Теория с опорой на источники
    \item Подробно-расписанное решение задачи
    \item Подборка из 3-5 уникальных задач по теме
    \item Подборка задач на эту тему из олимпиад
\end{enumerate}
\subsection{Апробация}
\begin{enumerate}
    \item К каждой теме будет добавлен тест из 5 вопросов в гугл-форме, по окончании которого будет высвечиваться результат прохождения и подсказки
    \item Чат в телеграме для обсуждения решений задач из моих подборок
    \item В конце пособия будет по одной контрольной задаче из каждой темы, которые я буду проверять и высылать свои комментарии. Проверка будет доступна аудитории в течение 1 месяца после публикации пособия
\end{enumerate}
\subsection{Содержание}
\begin{enumerate}
    \item Вспомогательные темы \begin{enumerate}
            \item Скрининг
            \item Разделяющее и объединяющее равновесие
            \item Возможно понадобится еще какие-то пояснения
        \end{enumerate}
    \item Модель Хотеллинга \begin{enumerate}
            \item Теория
            \item Практика
            \item Задачи для тренировки
        \end{enumerate}
    \item Сигналинг на рынке труда \begin{enumerate}
            \item Теория
            \item Практика
            \item Задачи для тренировки
        \end{enumerate}
    \item Вертикальная дифференциация товара \begin{enumerate}
            \item Теория
            \item Практика
            \item Задачи для тренировки
        \end{enumerate}
    \item Итоговый тест
\end{enumerate}
\subsection{Источники}
\begin{enumerate}
    \item Рэм Бахарев "Олимпиадная экономика"
    \item В.М.Гальперин "Микроэкономика"\space2 тома 1999
    \item Google Scholar (для поиска статей про вышеуказанные модели)
\end{enumerate}

\section{Чему мне придётся научиться в процессе выполнения проекта?}
\begin{enumerate}
    \item Улучшить свои способности искать информацию через google-scholar
    \item Изучить пакет tcolorbox для оформления страниц в LaTeX
    \item Изучить пакет pgfplots для создания графиков в LaTeX
    \item Научиться вводить более сложные математическикие формулы в LaTeX
    раньше
    \item Илучшить свое понимание этих тем и способов решения задач на эти темы
\end{enumerate}

\section{Взаимодействие с другими людьми в рамках выполнения проекта}
\begin{enumerate}
    \item Консультации с моим учителем экономики - А.Н.Челеховским
    \item Рассказ о своем учебном пособии на летнем курсе от Климента Шамаева и Даниила Перевалова, чтобы
    распространить его среди подходящей аудитории
\end{enumerate}

\section{Возможные риски и пути их преодоления}
\begin{enumerate}
    \item Вузовская математика в моделях в статьях из google-scholar. \textbf{Пути преодоления:} Использование упрощенных вариаций моделей, которые встречались на олимпиадах и выездных школах. В крайнем случае корректировка содержания пособия.
    \item Недостаток знания системы LaTeX - просмотр видеоуроков в интернете
\end{enumerate}

\section{Поэтапное планирование}
\begin{enumerate}
    \item Поиск статей про эти модели на Google Scholar
    \item Просмотр материалов про эти модели, которые встречались на выездных школах
    \item Определение точного состава вспомогательных тем
    \item Составления теоретической части
    \item Консультация с А.Н.Челеховским, по поводу найденной мной теоритической информации про эти модели.
    \item Написание подробного решения задач по каждой теме
    \item Сбор задач на эти темы с олимпиад
    \item Составление уникальных подборок с авторскими задачами по этим темам
\end{enumerate}

\section{Ход работы над проектом}
Я уже выполнил больше одной трети намеченной работы. Мой трекер активности насчитал суммарно 32 часа, на протяжении
которых я выполняю эту работу. По моим оценкам суммарно на выполнение потребуется затратить где-то 80 часов.
\begin{enumerate}
    \item Изучил дополнительную информации для работы в системе компьютерной верстки LaTeX
    \item Написал вступительную главу своего пособия
    \item Изучил пакет tikzpicture и pgfplots для создания иллюстраций и графиков в системе LaTeX, необходимых для 
    лучшего понимания темы
    \item Написал 1 из 3 основных содержательных блоков моего пособия, про сигналинг на рынке труда (Модель Спенса)):
    \begin{enumerate}
        \item Прочитал статью первоисточник
        \item Изучил материалы по данной теме от выездных школ (повторил материал от Луки Логинова, который
        рассказывал эту тему на Зимней Выездной Школе от Олмата, и материал от Ивана Дедюхина, который рассказывал
        эту тему на Интенсиве ВЭШ к ВП)
        \item Расписал теоритический блок
        \item Расписал предпосылки модели
        \item Описал возможные равновесия
        \item Расписал решение своих задач для примера, опираясь на типовые задачи, которые рассматривают выездные школы
    \end{enumerate}
\end{enumerate}

\section{Итоговый продукт}
Мое виденье готового продукта заключается в создании:
\begin{enumerate}
    \item Электронной версии пособия, которую я буду распространять
    \item Форума в телеграмм (чат с топиками) где я буду отвечать на возникающие вопросы и возможно расширять пособие
    \item После распространения (через тематические группы и чаты лицея) я буду проверять задания, которые мне будут загружать в гугл форму, в течении некоторого периода времени
    \item Планирую завершить работу над пособием и начать распространение (после консультаций по теоритической части
    и корректности) к началу сентября
\end{enumerate}
    
\section{Каково соотношение между поданной вами проектной заявкой и работой, которую вы представляете сейчас?}
    Да. На данный момент я продвигаюсь в соответствии со своим планом и концепцией, которую я определил для своего 
    пособия в момент написания первой заявки
    
\section{Информация о консультанте}
    Я уже связался с Александром Николаевичем Челеховским. Он задавал мне вопросы по поводу причин выбора мной данных
    тем. В дальнейшем я
    постараюсь консультироваться с ним по поводу хода работы над моим пособием
    
\end{document}
